%===================================================================================
% Chapter: Conclusions
%===================================================================================
\chapter*{Conclusiones}\label{chapter:conclusions}
\addcontentsline{toc}{chapter}{Conclusiones}
%===================================================================================

En la tesis se realizó una propuesta para el proceso de detección de intrusos en redes informáticas utilizando el conjunto NSL-KDD. En el proceso se analizaron los datos para entender mejor su comportamiento. Se observaron los resultados de los algoritmos bosques aleatorios y redes neuronales artificiales variando sus parámetros. En el caso del segundo se probaron 3 modelos diferentes. 

Como resultado final, se obtuvo una tasa de aciertos de 0.77 en el conjunto KDDTest+, 0.56 con KDDTest-21 y 0.98 con el conjunto creado a partir de la unión de los dos anteriores. Cabe destacar que el modelo resultante no es perfecto pero mantiene un valor por debajo del 2\% en la clasificación de los datos falsos positivos. La propuesta resultante es la aplicación del proceso de normalización de los datos, aplicar el algoritmo de bosques aleatorios para la eliminación de características que aportan muy poca información y para la fase de clasificación utilizar las redes neuronales.

% accuracy donde
% propuesta
    % normalizacion de los datos
    % random forest
    % aplicar rna para la clasificacion