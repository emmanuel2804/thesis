%===================================================================================
% Chapter: IDS
%===================================================================================
\chapter{Sistema de Detección de Intrusos}\label{chapter:ids}
% \addcontentsline{toc}{chapter}{Sistema de Detección de Intrusos}
%===================================================================================

Entre los años 1984 y 1986, Dorothy Denning y Peter Neumann desarrollaron un primer modelo de IDS \cite{denning1985requirements} para proteger los sistemas de cómputos. Pertenecen a la rama de la computación de las redes aunque en los últimos tiempos se han aplicado para su creación técnicas de distintas disciplinas como son algunas de inteligencia artificial. En la última década se han logrado avances en esta tecnología, obteniendo mejores resultados. A pesar de ello no se ha logrado un sistema con una efectividad lo suficientemente alta y aún es una línea de investigación activa. Se considera que estos sistemas tienen una gran importancia, ya sea económica o social, al proteger los datos delicados que viajan por la red y mantener la privacidad de cada usuario. Entre las técnicas de inteligencia artificial más utilizadas se encuentra el aprendizaje de máquinas.

\section{Problemas fundamentales}
Los IDS basados en técnicas de aprendizaje de máquinas son sistemas entrenados con un determinado conjunto de datos que son empleados más adelante en la detección de intrusos a una red. Una revisión general al proceso de filtrado de paquetes permite identificar los siguientes problemas:
\begin{itemize}
    \item Detección de comportamiento no observado.
    \item Porcentaje de detección de intrusos.
    \item Falsos positivos.
    \item Falsos negativos.
    \item Equipo final para su implementación.
    \item Tipo del ataque.
\end{itemize}

Esta investigación se centra en los problemas de los casos no observados [ref], falsos positivos [ref] y falsos negativos [ref], aplicados sobre el conjunto de datos NSL-KDD, determinando si un paquete de red posee un comportamiento normal o de un intruso. En otras investigaciones la clasificación se realiza en paquetes normales y distintos tipos de ataques como pueden ser denegación de servicios (DoS), Probe, ataque remoto a un equipo local (R2L), entre otros. Esto es una visión más general del problema aquí abordado, que de igual manera, es un problema de clasificación. Por la naturaleza de los paquetes y sus características, el problema se puede reducir a la búsqueda de patrones en datos de los que se saben sus clases y determinar la similitud entre alguno de estos y nuevos ejemplos.
    
% TODO:
Otra observación

Este problema ha sido analizado sobre el conjunto de datos NSL-KDD, que es una mejora el conjunto original KDDCup99, dando como resultado un gran número de investigaciones sobre el tema. A pesar de que existen herramientas para detectar intrusos en una red, muy pocas son gratis. Algunas de las soluciones han sido la aplicación de un conjunto de reglas en firewalls [reglas firewall] que monitorean el sistema pero las posibilidades de estos son escasas y dependen del conocimiento sobre los diferentes tipos de ataques y/o intrusos.

\section{Tipos de sistemas}
Existen varios tipos de IDS, pero para el propósito de este documento se pueden separar en basados en la red y basados en el host. En el caso de los basados en la red monitorean el tráfico de ella, analizan su actividad y la de los protocolos de aplicaciones en búsqueda de actividad sospechosa. Usualmente se colocan en los límites entre redes, como próximos a los cortafuegos de borde, routers, servidores de red privada virtual (VPN por sus siglas en ingles), servidores de acceso remoto o dispositivos de redes inalámbricos, entre otros. Los basados en el host monitorean las características de un solo equipo y los eventos que ocurren dentro de él. Pueden monitorear la red del equipo donde se encuentra el software, los registros del sistema, procesos en ejecución, aplicaciones activas, acceso y modificación de ficheros y aplicaciones del sistema que cambien la configuración. En su mayoría se implementan en equipos críticos como servidores con acceso público o servidores con información delicada.

Cada una de estas clasificaciones puede ser separada a su vez en las siguientes:

\begin{itemize}
    \item \textbf{Detección basada en firmas:} una firma es un patrón que corresponde a una amenaza conocida. Este proceso compara firmas en eventos observados para identificar posibles incidentes. Es muy efectivo detectando patrones iguales a los que almacena pero tienen un gran número de fallos cuando se trata de amenazas no registradas, disfrazadas con el uso de técnicas de evasión o variaciones de las ya conocidas.
    \item \textbf{Detección basada en anomalías:} es el proceso de comparar definiciones sobre cual actividad es considerada normal en los eventos observados para identificar desviaciones significantes. Utiliza una serie de perfiles que representan el comportamiento normal de varios elementos como usuarios, hosts, conexiones de red o aplicaciones. Por ejemplo, el perfil para una red empresarial en una hora normal de trabajo muestra que en la navegación web se consume el 13\% del ancho de banda total. Si el sistema detecta una diferencia considerable en este parámetro podría significar que la red se encuentra bajo una amenaza. Los perfiles pueden ser configurados para varios comportamientos, como el número de correos electrónicos enviados por un usuario, la cantidad de intentos fallidos al iniciar sesión por un host o el tiempo de uso de un proceso determinado. Una de las ventajas de este método es que es muy efectivo ante amenazas desconocidas.
    \item \textbf{Análisis de protocolo con estados:} compara perfiles predeterminados de definiciones generalmente aceptadas de actividad no maliciosa por cada uno de los estados de protocolos contra eventos observados para identificar desviaciones. A diferencia de la detección basada en anomalías, que utiliza perfiles de host o de redes específicos, esta se basa en perfiles universales desarrollados por el proveedor que especifica cuales protocolos y cuales no deben ser usados. La parte de estados en el nombre del método quiere decir que el sistema es capaz de entender, darle seguimiento al estado, transportar y tener una noción de los protocolos de aplicaciones. Por ejemplo, cuando un usuario inicia una comunicación por el protocolo de transferencia de archivos (FTP por sus siglas en ingles), el estado inicial es de no autenticado. En esta fase los usuarios solo deberían poder ejecutar unos pocos comandos como una petición de ayuda o enviar nombre de usuario y contraseña. Una parte importante de entender el estado es emparejar pedidos con respuestas, para cuando ocurra un intento de autenticación FTP, se pueda determinar si fue satisfactorio encontrando el código de confirmación en el mensaje de respuesta correspondiente al pedido. En caso de que el usuario complete la autenticación entonces podrá utilizar los comandos disponibles. Por otra parte, si se intenta utilizar alguno de estos en un estado que no se le ha confirmado la autenticación, se puede considerar como actividad sospechosa. Los sistemas de análisis de protocolo con estados pueden identificar una secuencia de comandos no esperados, en caso de que se necesite un orden específico y recordar si una sesión está autenticada o no. El análisis de protocolo en el nombre significa que realiza un chequeo exhaustivo de las características de los comandos individuales, como puede ser la longitud mínima y máxima de los argumentos.
\end{itemize}

\section{Fases de la detección y prevención de intrusos}
Los problemas tratados en esta tesis pueden ser enmarcados como sistemas de detección de intrusos, estos forman parte de los sistemas de detección y prevención de intrusos, los que están compuestos por las siguientes fases:

\begin{itemize}
    \item Almacenar información relacionada con los eventos observados. Usualmente la información es almacenada de forma local y puede ser enviada a otros sistemas como un servidor central encargado de archivar todos los eventos, información de seguridad, administradores de solución de eventos y sistemas de administración empresarial.
    \item Notificar a los encargados de la seguridad de un evento de importancia observado. Esta notificación, conocida como alerta, es enviada por cualquier vía de mensajes ya sea un correo electrónico, una página web o programas definidos por el usuario. La notificación normalmente incluye la información básica del evento. El administrador debe acceder al sistema para una mayor especificación.
    \item Elaborar un reporte. Los reportes resumen los eventos monitoreados y facilitan detalles sobre algún evento en particular de interés.
\end{itemize}

Los sistemas de prevención de intrusos se diferencian de los sistemas de detección de intrusos por una característica: los sistemas de prevención pueden responder en caso de la detección de una amenaza y prevenir que esta ocurra satisfactoriamente.

% TODO:
\section{Redes neuronales}
% Historia y muela

Las redes neuronales artificiales (RNA o ANN por sus siglas en inglés) son algoritmos de la rama de la inteligencia artificial que dado un conjunto de datos de ejemplo, ofrecen una función, sin importar el dominio de los valores de entrada y de salida, capaz de aproximar el resultado. Han brindado resultados excelentes en el reconocimiento de escritura a mano, palabras en audios y rostros.

Los primeros acercamientos a esta nueva tecnología fueron a principios de los a\~nos 1950 inspirados en los sistemas de aprendizaje biológicos. Son un conjunto de unidades simples, conectadas de manera densa entre ellas, donde cada unidad toma un real (que puede ser o no la salida de otra) y produce como resultado otro valor real (que puede convertirse en la entrada de otra unidad). El cerebro humano se estima que posee una red neuronal con $10^{11}$ unidades densamente conectadas, cada una de ellas conectadas con otras $10^{4}$ en promedio. La velocidad de comunicación entre una neurona y otra se encuentra en el orden de $10^{-3}$ segundos, por otra parte, las computadoras poseen una velocidad de $10^{-10}$ segundos. Los humanos son capaces de tomar complejas decisiones sorprendentemente rápido. Un ejemplo de esto es el reconocimiento de rostros, a un humano promedio le toma aproximadamente $10^{-1}$ segundo reconocer visualmente a su madre. Teniendo en cuenta la velocidad de comunicación entre neuronas, este proceso no excede unos cientos de pasos. Esto ha llevado a la comunidad científica a pensar que el sistema neuronal biológico es capaz de realizar tareas de procesamiento de información de manera altamente paralelas distribuidas en varias neuronas. Una de las motivaciones principales de es lograr un sistema capaz de hacer procesos altamente paralelos en representaciones distribuidas \cite{brown1990learning,churchland1992computational,mitchell1997machine}.

Después de los primeros pasos en la creación de las redes neuronales, los científicos se encontraron con el problema de entrenar grandes arquitecturas de una forma eficiente. A mediados de los a\~nos 1980, varios estudios de diferentes partes del mundo, descubrieron de forma independiente el algoritmo de propagación hacia atrás, el cual utiliza el método de optimizaci\'on de descenso por gradiente para entrenar cadenas de operaciones param\'etricas. Unos de los pioneros en utilizar estas técnicas con grandes éxitos fueron los laboratorios Bell, cuando Yann LeCun utilizó las ideas iniciales de las redes convolucionales y la propagación hacia atrás. Como resultado obtuvieron una red que podía reconocer dígitos escritos a mano con alta efectividad, la cual nombraron LeNet \cite{lecun1995comparison}, y fue utilizada por el Servicio Postal de los Estados Unidos en la decada de 1990. A pesar de que las redes neuronales artificiales fueron inspiradas por el sistema biológico, ambas poseen características que las diferencian.

% diseño basado en hardware

\subsection{Funcionamiento}
% funcion de red
% el aprendizaje
% la eleccion de una funcion de costo
% paradigma de aprendizajes
    % supervisado
    % no supervisado
    % por refuerzo
% tipo de entrada

% algoritmos de aprendizaje

\subsection{util para ids}
% las ann son especiales en casos de:
    % la entrada es representada por varios pares de atributos
    % la salida de la funcion objetivo puede ser de cualquier tipo
    % los ejemplos de entrenamiento pueden contener errores
    % largos tiempos de entrenamiento son aceptables
    % se requiere de una rapida evaluacion de la funcion objetivo aprendida
    % no es importante que los humanos entiendan la funcion objetivo aprendida

% aplicaciones

% potencia de calculo

% critica

% clases y tipos

\section{Conjunto de datos y características}
Distintas investigaciones se han enfocado en realizar sistemas de detección de intrusos teniendo como fuente de datos NSL-KDD. En 1998 la Agencia de Proyectos de Investigación Avanzados de Defensa, más conocida por su acrónimo DARPA, recopiló y almacenó la información del tráfico de una red que nombró KDDCup99 llegando a ser uno de los conjuntos de datos públicos que contiene diferentes ataques vigentes en la actualidad y más utilizados por la comunidad científica. NSL-KDD fue diseñado para resolver varios de los problemas existentes en el conjunto original como son la información redundante, tanto en los conjuntos de entrenamiento como de prueba, y la cantidad de datos en cada subconjunto es razonable, lo cual hace comparable y consistentes los resultados de cualquier trabajo y evita tener que escoger un subconjunto aleatorio para los diferentes procesos.

Una aproximación se basa en la creación de reglas basadas en las características de los datos. Además, se propone el uso de algoritmos genéticos para la búsqueda del conjunto de reglas ideales. Igualmente, se han utilizado técnicas basadas en sistemas inmunes artificiales, seleccionando varios equipos de la red para tareas diferentes como son la recolección de información y eliminación de tráfico malicioso.

Otro de los estudios sobre la detección de intrusos, para determinar el tipo de los paquetes utiliza variantes del algoritmo árboles de decisión mostrando resultados mejorados. También se han aplicado los bosques aleatorios (en ingles, random forest) ofreciendo clasificaciones aceptables. Otro de los algoritmos de aprendizaje de máquina utilizado es k vecinos cercanos, agrupando en subconjuntos los datos para determinar su clasificación.

Una de las técnicas más utilizadas son las redes neuronales que han tenido gran auge en los últimos tiempos. Sin embargo en la literatura consultada no se evalúa la efectividad del sistema ante casos que nunca ha visto y por el rápido avance de las tecnologías esto es de suma importancia.

\subsection{Propuesta de Proceso de Detección de Intrusos en el conjunto de datos}
Determinar si un paquete es malicioso, es una problemática que involucra diferentes procesos y alternativas de solución. Encontrar una combinación de algoritmos que se ajuste a las características particulares de este problema, permitiría establecer una propuesta general.

La solución que se propone para el proceso de detección de intrusos en NSK-KDD se divide en 3 etapas:

\begin{itemize}
    \item Procesamiento o normalización de los datos
    \item Reducción de dimensiones o eliminación de características no relevantes.
    \item Clasificación.
\end{itemize}

En cada etapa pueden ser utilizados diferentes algoritmos dependiendo de las características del problema a resolver en cada una de ellas. Para esto es necesario identificar que algoritmos son los más prometedores así como la mejor combinación de ellos.

% TODO:
\subsection{Arboles aleatorios}
explicación del funcionamiento del algoritmo y como mejora los resultados en este problema[ver tesis de RNN en IDS]