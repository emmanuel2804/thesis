%===================================================================================
% Chapter: Introduction
%===================================================================================
\chapter*{Introducción}\label{chapter:introduction}
\addcontentsline{toc}{chapter}{Introducción}
%===================================================================================

Hoy en día las redes de computadoras han cobrado un gran auge, el ejemplo más latente es el propio internet. Su crecimiento es tan alto que se estimó que en 2019 el 53.6\% de la población mundial tenía acceso a este servicio \cite{internetAcces}. La informatización es un campo que ha ido de la mano junto a las redes, llevando al mundo digital servicios como los bancarios, la telefonía, el correo, entre otros muchos. Con la unión de ellos es posible que personas que están muy lejos geográficamente puedan intercambiar cualquier información o realizar pagos sin la necesidad de tener dinero físico. Es evidente la importancia del contenido que viaja por las redes en nuestros tiempos y su delicadeza por lo que la seguridad es primordial.

Por el rápido crecimiento del internet y la alta demanda por parte de la población, existen diversos equipos para la expansión de este. Esta diversidad causa muchas vulnerabilidades que pueden poner en riesgo la información que viaja por las redes. También la propia estructura descentralizada del internet implica que no exista un alto control sobre este y ello atenta contra la seguridad de los usuarios. Por estas razones cada día surgen nuevas amenazas lo que dificulta su detección.

Los expertos están al tanto de estos problemas y son muchas las empresas que trabajan para eliminar o disminuir las vulnerabilidades existentes. Por lo antes mencionado, no es sencillo enfrentar el problema debido a su estructura. Al poseer un conjunto infinito de posibles errores y formas de ataque, no se puede resolver el problema con algoritmos que no puedan adaptar a nuevas condiciones. Entre las soluciones, han surgido ideas que utilizan los algoritmos más novedosos de inteligencia artificial.

Una de las formas más efectivas de luchar contra los ataques en redes es colocando un Sistema de Detección de Intrusos \cite{scarfone2012guide} (IDS por sus siglas en inglés, Intrusion Detection System) junto con un firewall \cite{wack2002guidelines}, uniendo la detección inteligente de uno y el poder de bloqueo del otro. En dependencia del tipo de red el sistema puede ser colocado en hardwares especializados, como switch, routers, junto al equipo encargado de realizar la función de puerta de enlace del sistema, entre otros; debido a que por estos puntos pasa la mayoría de la información de la red \cite{tanenbaum1996sistemas}. En el caso de los hardwares especializados, se tiene equipos con bajo poder de cómputo que tiene que analizar a gran velocidad todo el tráfico de la red, obligando a que el software sea lo más ligero y eficiente posible.

Los datos delicados que viajan por las redes necesitan de una elevada protección, de no llevarse a cabo un proyecto a gran escala que incorpore algoritmos inteligentes para la detección de intrusos pueden ocurrir daños irreversibles a la economía, la sociedad, entre otros aspectos de nivel mundial. Un ejemplo de esto fue el ataque realizado en noviembre de 2009, donde un grupo hackers robó alrededor de 9.4 millones de dólares en menos de un día \cite{harper2011gray}. No se conoce cuál es la próxima amenaza ni que tan grave pueda ser. Detener a un atacante antes de que comprometa la seguridad de toda una red es clave, más cuando se viven tiempos en los que las redes informáticas tienen gran relevancia en el dia a dia de cada ser humano. Algunas de las ventajas de tener el IDS adecuado es evitar costos innecesarios, crear parches de seguridad a tiempo, no teniendo que cambiar la estructura de la red para su funcionamiento, entre otros. Debido a la utilidad evidente de este tipo de proyectos se realizó una búsqueda en el Instituto de Criptografía de la Universidad de la Habana sobre trabajos de esta envergadura y no existen registros que indiquen que ya se ha realizado alguno con anterioridad. En particular, el IDS que aquí se propone utiliza una serie de algoritmos de aprendizaje de máquinas (\textit{machine learning}). En su fase de entrenamiento requiere un alto poder de cómputo pero a la hora de ejecutar su trabajo es muy efectivo y eficiente.

\section*{Problema general}
A partir de todas las dificultades expuestas, haciendo énfasis en la carestía de un sistema como el antes mencionado, se propone para esta investigación el siguiente problema científico:
¿Se podrá obtener un modelo, basado en algoritmos de aprendizaje de máquina, capaz de detener todos los posibles intrusos de una red o una cantidad lo suficientemente buena?
Siendo el objeto de estudio: Sistema de detección de intrusos utilizando redes neuronales.

\section*{Problemas específicos}
\begin{itemize}
    \item Desproporción en el conjunto de datos. Si existen más datos de un tipo que de otro los resultados pueden inclinarse hacia los de mayor cantidad.
    \item Sobreajuste. Cuando los algoritmos de aprendizaje de máquina no mejoran sus resultados y las métricas de validación comienzan a degradar tras varias iteraciones sobre el conjunto de entrenamiento.
    \item Desajuste. Al comienzo del entrenamiento los algoritmos de aprendizaje de máquina obtienen bajos valores de validación y pueden aprender aún más de los datos.
    \item Largos tiempos de entrenamiento. Los algoritmos de aprendizaje de máquina requieren de grandes cantidades de cálculos para su entrenamiento por lo que si no se tiene un \textit{hardware} especializado este proceso se puede tardar.
\end{itemize}
% Cuando se utilizan algoritmos de aprendizaje de máquinas, en algunos casos, se pueden obtener resultados falso positivo o falso negativo; para este problema en específico, un ejemplo sería que el sistema seleccione un paquete de red como intruso y que no lo sea o viceversa, respectivamente. Cada autor, según sus necesidades, decide darle mayor importancia a uno de estos problemas o tratarlos por igual. Un ejemplo puede ser si el sistema se está implementado en una red que navega información de alta seguridad, es mejor que no pasen paquetes, aunque no sean intrusos, antes de que pase alguno que si lo sea. Existen varios algoritmos que se pueden aplicar por lo que la elección de uno no es tarea fácil; cada cual tiene sus debilidades y ventajas por lo que escoger una métrica para medir el más correcto no es trivial.

\section*{Hipótesis}
El contenido que viaja por las redes informáticas posee diferentes patrones de comportamiento, que se pueden clasificar en intrusos o usuarios legítimos. Por ello, este trabajo intenta crear estos patrones utilizando análisis estadístico para separar a usuarios de agentes externos maliciosos.

\section*{Objetivos generales}
Para la solución del problema científico se propone como objetivo principal de esta investigación utilizar métodos de aprendizaje de máquinas en un IDS y explorar los resultados obtenidos en diversos experimentos para escoger el mejor modelo capaz de clasificar la mayor cantidad correcta de paquetes que viajan por la red, siendo el campo de acción las redes neuronales y la inteligencia artificial.

% \section*{Objetivos específicos}
Y como objetivos específicos:
\begin{itemize}
    \item Entrenar varios modelos con un conjunto de datos ya definido.
    \item Analizar los resultados de los métodos estadísticos.
    \item Validar si existe algún grado significativo en los resultados.
    \item Comparar los resultados obtenidos con los de las bibliografías consultadas.
\end{itemize}

% \section*{Limitaciones}
Los límites de la investigación son las siguientes:

\begin{itemize}
    \item Redes neuronales: a finales de la década de los años 90 han cobrado mucho auge por sus relevantes resultados, muchos problemas se han atacado con ellas y hoy en día siguen en constante desarrollo. Se ha llegado al punto de que con las herramientas correctas se pueden crear de forma muy rápida y sencilla. Este trabajo va a profundizar en su variante más básica empleada en un IDS probando diferentes parámetros y evaluando los resultados. Como meta principal se propone obtener mejores resultados que los brindados por una línea base como es un clasificador totalmente aleatorio. 
    \item Conjunto de datos NSL-KDD: el proceso va a contar con estos datos y una variante del mismo para una mejora de los resultados. También explora las características de los datos en búsqueda de elementos innecesarios que puedan afectar el proceso de aprendizaje o el resultado final a través del algoritmo de bosques aleatorios.
\end{itemize}

% \section*{Situación problemática}
% TODO: pendiente justificación de que no hay proyectos como este en cuba
% Debido a que no puedo investigar nada de eso me limité al IC de la UH
% Debido a la utilidad evidente de este tipo de proyectos se realizó una investigación por los principales centros de investigación que se dedican a las redes de computadoras y a la inteligencia artificial en la Habana, Cuba, y no existen registros de un proyecto como este[pendiente justificación]. 

Con vista a cumplir los objetivos de la investigación y validar la hipótesis planteada se formulan las siguientes tareas de investigación:

\begin{itemize}
    \item Búsqueda y análisis de la literatura, artículos, materiales nacionales y extranjeros.
    \item Estudio del conjunto de datos para una mejor comprensión del comportamiento de los mismos.
    \item Diseño y elaboración de los modelos de sistemas.
    \item Entrenamiento y validación de los algoritmos en la detección de intrusos.
    \item Diseño y desarrollo de los experimentos para confirmar la validez de la hipótesis.
    \item Comparación de los resultados obtenidos utilizando métricas populares.
    \item Elaboración del informe final (tesis).
\end{itemize}

% \section*{Estructura de la Tesis}
La tesis se encuentra estructurada de la siguiente forma: introducción, tres capítulos, conclusiones, recomendaciones y bibliografía. 
En el capítulo 1 se brinda una explicación sobre los sistemas de detección de intrusos, las distintas fases que comprenden a los mismos y los diferentes tipos que existen. También se explica el funcionamiento de las redes neuronales y porque son útiles en este tipo de sistemas, además se da un breve bosquejo por el conjunto de datos sobre el que se va a estar trabajando, la propuesta planteada en este documento y el algoritmo de bosques aleatorios.

En el capítulo 2 se profundiza más en el conjunto de datos. Se explica su procedencia y se brinda una descripción detallada sobre las distintas partes que forman el conjunto de datos. También se analizan la diferentes transformaciones realizadas para mejorar los resultados del sistema final. En el capítulo 3 se muestran todos los resultados obtenidos a lo largo del proceso de perfeccionamiento del sistema acompañado de gráficas y tablas que validan los procesos. Para finalizar se presentan las conclusiones, recomendaciones y referencias bibliográficas consultadas durante la investigación.

% ------------------------- Aclaraciones del tutor----------------------- 

% la necesidad no queda clara porque no se menciona que tuviésemos carestía en esa area, lo que queda claro es que este tipo de trabajos son buenos y útiles pero no se sabe si aquí hay alguno/s ya hecho, si hay donde se hizo o no hay, la investigación que se hizo sobre este particular debe quedar reflejada en este documento para que el que lo lea vea el trabajo(esfuerzo investigativo) que se hizo, que lugar es el que lo solicita. Por otra parte la carencia ponerla al final de la introducción, esto da pie a la situación problemática y de ahi se desglosa todo lo demás. Te propongo reflejar los resultados de la búsqueda que se realizo sobre cubanos principalmente(para los antecedentes) y extranjeros que hayan hecho algo al respecto y después plantear la carencia: que en el instituto de criptografía de la uh no existe ninguna investigación, ni software ni nada al respecto, y esta es la necesidad que nos llevo a hacer este trabajo. de todos modos voy a puntualizar con el jefe del IC para tener la certeza de que no hay nada al respecto, esto es importante porque tu vas a defender la tesis por el IC.

% mover esto al capitulo 2
% Existen diversos softwares para la recopilación de los datos que van ayudar en esta tarea y es fácil obtener un gran conjunto de datos (aunque no de calidad óptima). 

% capitulo 3, particularidades del problema
% Se utilizará como lenguaje de programación Python por su cualidad de software libre [referencia] y por ofrecer diversos módulos con la mayoría de los algoritmos implementados lo que evita ser redundante.

% Es importante que en el documento quede esclarecido de donde se obtuvieron los datos para hacer el entrenamiento e investigar sobre la fiabilidad de los mismos. Ya vi que pusiste softwares pero debes mencionarlos y una breve descripción de los mismos sobre todo lo que hacen. Entonces propongo que un epígrafe en este capitulo expliques en síntesis estos softwares y en el segundo digas lo que quisiste decir en el párrafo anterior, o sea, de los softwares que pusiste cual seleccionaste y porque para tu trabajo. El párrafo anterior para el siguiente capitulo.