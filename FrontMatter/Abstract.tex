%===================================================================================
% Chapter: Abstract
%===================================================================================
\chapter*{Resumen}\label{chapter:abstract}
% \addcontentsline{toc}{chapter}{Análisis del conjunto de datos}
%===================================================================================

% que es ids
% que es NSL-KDD
% que se utilizo
% objetivos
% etapas para conseguir los objetivos
% que se hace en cada una
% algoritmos utilizados
% resultados obtenidos
% 

Los sistemas de detección de intrusos son una herramienta utilizada en las redes informáticas para prevenir ataques a las mismas. El conjunto de datos NSL-KDD es un ejemplo del tráfico que viaja por estas y contiene una gran variedad de ataques. Como solución se propone utilizar las redes neuronales artificiales en la clasificación de los datos en tráfico común o anómalo. Este trabajo tiene como objetivo encontrar los mejores parámetros de los algoritmos propuestos para el problema de detección de intrusos en el conjunto  NSL-KDD. Para esto se dividió el proceso en las etapas de normalización de los datos, reducción de características y clasificación. En el caso de la segunda se realiza un análisis de la importancia de cada característica en la clasificación con el algoritmo de bosques aleatorios. Los resultados obtenidos por la propuesta final alcanzan una tasa de acierto de 77\% ante datos poco analizados en la fase de entrenamiento. Estos resultados demuestran el potencial de las redes neuronales como parte de los sistemas de detección de intrusos en una red.